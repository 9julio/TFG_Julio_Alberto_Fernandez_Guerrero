\chapter{Descripción de competencias}

\begin{table}[hp]
  \centering
  \label{tab:tec-especifica}
  %\zebrarows{1}
  \begin{tabular}{p{0.6\textwidth}}
    \hline
    \rowcolor{gray!25} Tecnologías de la Información \\
    \hline
    Computación \\
    \hline
    \rowcolor{yellow} \textbf{Ingeniería del Software} \\
    \hline
    Ingeniería de Computadores \\
    \hline
  \end{tabular}
    \caption{Tecnología específica cursada por el alumno}
\end{table}

\newpage

\begin{center}
\rowcolors{1}{gray!25}{white}
\begin{longtable}{p{.40\textwidth} p{.60\textwidth}}
\hline \hline
  \textbf{Competencia} & \textbf{Justificación} \\
    \hline \hline
    Capacidad para desarrollar, mantener y evaluar servicios y sistemas software que satisfagan todos los requisitos del usuario y se comporten de forma fiable y eficiente, sean asequibles de desarrollar y mantener y cumplan normas de calidad, aplicando las teorías, principios, métodos y prácticas de la Ingeniería del Software. & Este trabajo consiste en la implantación de un entorno de integración continua que garantice un alto grado de automatización de las pruebas, de manera fiable y confiable, debido a que se prevé que este entorno de integración continua trabaje con tareas críticas, y dado que se deben obtener los mismos resultados ejecutando los tests mediante el entorno de integración continua que si se realizaran de manera manual, se justifica que se hará uso de la capacidad de desarrollar, mantener y evaluar servicios y sistemas software que satisfagan todos los requisitos del usuario y se comporten de forma fiable y eficiente, que sean asequibles tanto en su desarrollo como en su mantenimiento, y que apliquen normas de calidad mediante teorías, principios y prácticas de la Ingeniería del Software.\\
    \hline\hline
    Capacidad para valorar las necesidades del cliente y especificar los requisitos software para satisfacer estas necesidades, reconciliando objetivos en conflicto mediante la búsqueda de compromisos aceptables dentro de las limitaciones derivadas del coste, del tiempo, de la existencia de sistemas ya desarrollados y de las propias organizaciones. & Se requiere analizar y comprender la metodología propia de desarrollo utilizada en Madrija, dado que se debe incorporar en la misma un entorno de integración continua. Esta modificación a la metodología, podrá suponer cambios que generen conflictos y que deban ser resueltos considerando un equilibrio entre los objetivos buscados con la integración continua y la metodología existente. Se justifica así la capacidad indicada.\\
    \hline \hline
    Capacidad de identificar y analizar problemas y diseñar, desarrollar, implementar, verificar y documentar soluciones software sobre la base de un conocimiento adecuado de las teorías, modelos y técnicas actuales. & Dadas las necesidades software que se deben satisfacer,  y la aparición de conflictos durante la implantación del entorno de integración continua, se deben implementar, verificar y documentar las solucionas software que han sido desarrolladas, para resolver los conflictos surgidos, durante el desarrollo del proyecto para su correcta implantación, justificando así que mediante la implantación del entorno de integración continua, se adquiere la capacidad de identificar y analizar problemas y diseñar, así como de desarrollar, implementar, verificar y documentar soluciones software sobre la base de un conocimiento adecuado de las teorías, modelos y técnicas habituales.\\
    \hline \hline
    Capacidad de dar solución a problemas de integración en función de las estrategias, estándares y tecnologías disponibles. & Mediante la implantación e integración del entorno de integración continua en la metodología de desarrollo de \ac{Madrija} junto con todas sus herramientas, Vagrant o VirtualBox entre otras muchas, siendo necesaria la capacidad para dar solución a problemas de integración en función del conocimiento de estrategias, estándares y tecnologías disponibles.\\
    \hline \hline
    \caption{Justificación de las competencias específicas abordadas en el TFG}
\end{longtable}
\end{center}