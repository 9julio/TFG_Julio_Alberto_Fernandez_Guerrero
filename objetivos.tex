\chapter{Objetivos}
\label{chap:objetivos}

\drop{D}ado el enfoque y las ideas presentadas en la introducción, este capítulo versa sobre el objetivo principal de este trabajo, así como los objetivos parciales que se pretenden alcanzar durante el desarrollo del mismo. Se definen así una serie de hitos a alcanzar para conseguir el objetivo final.

\section{Objetivo principal}

\fbox{\parbox[b]{\linewidth}{\hspace{1em} El objetivo de este trabajo es analizar, diseñar e implementar un entorno de integración continua, que se adapte a la metodología, formas y prácticas de desarrollo que tiene \ac{Madrija}.}}

Se desea destacar que este proyecto no versa sobre el desarrollo de un producto, ni tampoco sobre el desarrollo de software.

%Se procede a desglosar el objetivo principal en objetivos parciales.

Los subobjetivos a tratar en este trabajo son:

\begin{itemize}
\item Analizar y determinar herramientas de integración continua.
\item Desarrollar la infraestructura para el sistema de integración continua.
\item Adaptarse a la metodología de desarrollo.
\item Alcanzar el máximo grado de automatización posible de los tests.
\item Proponer mejoras.
\end{itemize}

\section{Objetivos parciales}

\subsection{Analizar y determinar herramientas de integración continua}
\ac{Madrija} es consciente de los beneficios de implantar un entorno de integración continua, por ello, requiere que se realice un estudio para analizar las herramientas y elegir la que más beneficie y mejor se adapte a la metodología de desarrollo de la empresa.

\clearpage

\subsection{Desarrollar la arquitectura e infraestructura para el sistema de integración continua}
\ac{Madrija} no cuenta con un sistema de integración continua por lo que se deberá analizar, diseñar y desarrollar una arquitectura e infraestructura para el sistema de integración continua.

\subsection{Adaptarse a la metodología de desarrollo}

\ac{Madrija} cuenta con una metodología propia de desarrollo, que será mejorada y complementada por la implantación del sistema de integración continua, para ello, se debe analizar y comprender la metodología utilizada por \ac{Madrija}.

\subsection{Alcanzar el máximo grado de automatización posible de los tests}

\ac{Madrija} realiza de manera manual, utilizando Maven, las pruebas a sus proyectos lo que supone pérdidas de tiempo, además, también influye el factor humano dado que existe la posibilidad de que no se realicen dos pruebas iguales a un mismo desarrollo en un mismo test o, incluso, que haya días en los cuales no se realicen pruebas.

\subsection{Proponer mejoras}

Para la implantación del entorno de integración continua, \ac{Madrija}, deberá realizar una serie de cambios dentro de su metodología para que dicha implantación sea factible, es decir, se debe analizar la forma en la que desarrolla \ac{Madrija} y estudiar la manera más fácil mediante la cual realizar dicha implantación.

\section{Justificación}

Este trabajo tiene varias causas que lo justifican entre las que destacan, el ahorro de tiempo, como se ha dicho antes, \ac{Madrija} realiza las pruebas a sus proyectos de manera manual, utilizando Maven, software mediante el cual se orquestan una parte de las pruebas, pero necesita ser ejecutado por el equipo de desarrollo, por lo que cabe la posibilidad de que sucedan fallos humanos a la hora de realizar la ejecución de dichas pruebas, ocasionando que se cometa el error de no realizar siempre las mismas pruebas al mismo desarrollo.

Se hace necesario un análisis a fondo de la forma de desarrollo de la empresa para que la implantación del entorno de integración continua sea lo más factible posible.

