\chapter{Resumen}

Hoy en día, muchas empresas desarrollan software desde un enfoque clásico pero a la vez intentan adaptarse a los nuevos tiempos intentando introducir las nuevas metodologías en sus proyectos, en especial, las metodologías ágiles. En el enfoque clásico que utilizan las empresas, el desarrollador tan pronto desarrolla como ejecuta tests unitarios al software, es decir, muchas empresas aún no conocen la importancia de separar el trabajo y, sobre todo, el ahorro en costes y tiempo que supone hacer esto.\\

Madrija Consultoría, S.L. no es una excepción, dado que aplica una metodología propia para gestionar el ciclo de vida de sus desarrollos pero los desarrolladores actúan tanto de desarrolladores como de testers, por lo que pretende introducir una nueva práctca de las metodologías ágiles para mejorar su metodología, en concreto, nacida de la mano de Martin Fowler, la integración continua, surgió con el objetivo de facilitar el trabajo en equipos de desarrollo y automatizar las tareas de integración. La integración continua se basa en la construcción automática de proyectos con frecuencia alta, promoviendo la detección de errores en un momento temprano para poder dar prioridad a corregir dichos errores.\\

Por lo tanto, se hace necesaria la implantación de un sistema de integración continua dentro de la empresa para alcanzar el mayor grado de automatización posible en sus pruebas de proyectos permitiendo así a los desarrolladores ahorrar tiempo en estas tareas para centrarse en otras, surgiendo así la necesidad de realizar un análisis, diseño e implantanción de un sistema de integración continua dentro de la empresa.\\

Por ello, surge la motivación de desarrollar este trabajo fin de grado, dentro del convenio FORTE, con  Madrija Consultoría, S.L.\\

\chapter{Abstract}

Nowadays, many companies develop software from a classical approach but now they try to adapt it to the new times, they are trying to introduce the new methodologies in their projects, especially the agile methodologies. In the classic approach, the developers do both things develop and executes unit tests, in other words, many companies still don't know the importance of separating the work and, above all, the savings in the costs and the time that means to do this.\\

Madrija Consultoría, S.L. isn't an exception, since it applies its own methodology for live cicle managment but the developers act as much of the developers as of the testers, but they are trying to introduce a new practice of the agile methodologies to improve its methodology, in this case, continuous integration, this practice borned with the aim of facilitating work in development teams and automate integration tasks. The continuous integration is the automatic construction of projects with high frequency, promoting the detection of errors in an early moment to give priority to correct these errors.\\

Therefore, it is necessary to implement a system of continuous integration within the company to achieve the highest degree of automation possible in their project tests thus allowing developers to save time in these tasks to focus on others, arising in the need to perform an analysis, design and implementation of a system of continuous integration within the company.\\

For this reason, the motivation to develop this end-of-grade job, within the FORTE agreement, with Madrija Consultoria, S.L.\\