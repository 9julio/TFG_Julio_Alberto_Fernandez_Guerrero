\chapter{Conclusiones}

\drop{E}n este último capítulo se determina si se han conseguido los objetivos parciales detallados en el segundo capítulo de este documento, incluyendo el objetivo principal también.

Para concluir, se presentan algunas propuestas de trabajo futuro realacionadas con la temática sobre la que versa este trabajo, se incluye al final una pequeña opinión personal del autor.

\section{Análisis de los objetivos del proyecto}
El objetivo principal de este trabajo era analizar, diseñar e implementar un entorno de integración continua, que se adapte a la metodología, formas y prácticas de desarrollo que tiene \ac{Madrija}. El objetivo ha sido cumplido mediante la implantación de un sistema de integración continua dentro de la empresa así como su instalación y configuración dentro de la arquitectura que fue diseñada para dicho propósito, obteniendo así la implantación del sistema de integración continua y su arquitectura en el servidor de la empresa.

Por lo tanto, el objetivo principal de este trabajo ha sido conseguido y a su vez los objetivos parciales, todo ello documentado a lo largo de este documento.

\begin{center}
\begin{longtable}{p{.80\textwidth} p{.20\textwidth}}
\hline \hline
  \rowcolor{gray!25}\textbf{Objetivo} & \textbf{¿Conseguida?} \\
    \hline \hline
    Analizar, diseñar e implementar un entorno de integración continua, que se adapte a la metodología, formas y prácticas de desarrollo que tiene \ac{Madrija} & \begin{center} \includegraphics[width=1cm]{tick.png} \end{center}\\
    \hline \hline
    \caption{Objetivo principal}
\end{longtable}
\end{center} 

\newpage

\begin{center}
\rowcolors{1}{gray!25}{white}
\begin{longtable}{p{.40\textwidth} p{.60\textwidth}}
\hline \hline
  \textbf{Objetivo} & \textbf{Justificación} \\
    \hline \hline
    Analizar y determinar herramientas de integración continua. & Se justifica haber superado este objetivo parcial en la iteración 1, debido a la prueba de concepto inicial realizada con Jenkins y superada con éxito, se elige a Jenkins como la herramienta que será utilizada para desarrollar el sistema de integración continua.\\
    \hline\hline
    Desarrollar la infraestructura para el sistema de integración continua. & Se justifica haber superado este objetivo parcial en la iteración 2, dada la aprobación por parte de \ac{Madrija} del prototipo de infraestructura presentado a la empresa para el desarrollo del sistema de integración continua.\\
    \hline \hline
    Adaptarse a la metodología de desarrollo. & Se justifica haber superado este objetivo parcial en las iteraciones 1 y 3, debido a la prueba de concepto que demostró la capacidad de adaptación de Jenkins y en la iteración 3, debido a la propuesta de cambios para mejorar su metodología de gestión del ciclo de vida de sus desarrollos.\\
    \hline \hline
    Alcanzar el máximo grado de automatización posible de los tests. & Se justifica haber superado este objetivo en las iteraciones 1 y 4, dado que en la primera iteración como se describe en el documento se consiguen automatizar tests con Maven e incorporarlos en Jenkins, y en la iteración 4, se incorpora a Jenkins ``Renombrator'', tests que realizaba la empresa de manera manual y que ahora se realizan de manera automática por parte de Jenkins cuando detecta algún cambio en el código.\\
    \hline \hline
    Proponer mejoras. & Se justifica haber superado este objetivo parcial dado que en la iteración 3, se proponen mejoras y cambios en la metodología propia de la empresa para gestionar el ciclo de vida de sus desarrollos.\\
    \hline \hline
      \rowcolor{white}\caption{Análisis de los objetivos parciales}
\end{longtable}
\end{center}

\newpage

\section{Trabajo futuro}

Tras finalizar este trabajo surgen nuevas cuestiones para un trabajo futuro entre las que destacan:

\begin{itemize}
\item \textbf{Solucionar el problema del plugin de GitLab:} Para poder realizar otra estrategia para realizar las ejecuciones del sistema de integración continua.
\item \textbf{Implantación de SonarQube:} Para controlar y medir la calidad del código.
\item \textbf{Mejorar la seguridad:} El sistema de integración está únicamente protegido por medio de \textit{``usuario y contraseña''} por lo que se podría aportar algún tipo de seguridad extra.
\end{itemize}

\section{Opinión personal}

La realización de este trabajo me ha servido para adquirir nuevos conocimientos tanto teóricos sobre integración continua y calidad del código fuente desarrollado, como aspectos tecnológicos (configuración de máquinas virtuales, sistemas UNIX, Jenkins, gestión y construcción de proyectos, etcétera).

Durante estos meses he adquirido conocimientos relacionados con integración continua y he reforzado otros conceptos de aspectos tecnológicos.

Concluido este trabajo fin de grado, debo decir que no es un punto y final. El desarrollo de este trabajo y el paso de las dificultades, me han sido útiles para saber qué únicamente conozco una pequeña parte de este gran mundo, la informática, y, por ello, seguiré en formación y buscando nuevos retos que me apasionen tanto o más como la realización de este trabajo.

Por lo tanto, como resultado global de este trabajo me encuentro muy satisfecho y con ganas de seguir mejorando, evolucionando e innovando en este mundo.\\ \\ \\ \\ \\ \\

\begin{flushright}
Ciudad Real, a \today\\
Fdo. \textit{Julio Alberto Fernández Guerrero}
\end{flushright}
